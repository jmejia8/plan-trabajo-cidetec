\documentclass[12pt,letterpaper, xcolor=table, x11names]{article}
%%%%%%%%%%%%%%%%%%%%%%%%%%%%%%%%%%%%%%%%%%%%%%%%%%%%%%%%%%%%%%%
%       Paquetes Necesarios
\usepackage[left=2cm,right=2cm,top=2cm,bottom=2cm]{geometry}
\usepackage{courier} \renewcommand{\familydefault}{\sfdefault}
\usepackage[utf8]{inputenc}
\usepackage[spanish]{babel}
\usepackage{amsmath, amsfonts, amssymb, titlesec}
\usepackage{fancyhdr}
\usepackage[x11names, table]{xcolor}
%%%%%%%%%%%%%%%%%%%%%%%%%%%%%%%%%%%%%%%%%%%%%%%%%%%%%%%%%%%%%%%
%   Definiciones de colores
\definecolor{gray}{RGB}{216, 216, 216}
\definecolor{gray2}{RGB}{231, 231, 231}
\definecolor{black}{RGB}{0, 0, 0}
%%%%%%%%%%%%%%%%%%%%%%%%%%%%%%%%%%%%%%%%%%%%%%%%%%%%%%%%%%%%%%%
%   Nuevos Comandos
\renewcommand{\labelenumi}{\roman{enumi}$)$}
\newcommand{\tache}{$\boldsymbol{\times}$}
%%%%%%%%%%%%%%%%%%%%%%%%%%%%%%%%%%%%%%%%%%%%%%%%%%%%%%%%%%%%%%%

% \date{Última actualización: \today}

\newcommand{\mytitle}{ %
% 
	\vspace*{2cm}
	\begin{center}
		{\LARGE \bf Estancia de Investigación}\\[0.1cm]%
		{\Large \bf Plan de Trabajo}\\[0.4cm]%
		{\large Fecha de elaboración: \today}
	\end{center}
	\vspace{0.7cm}
	% 
}

\begin{document}

\mytitle

El presente documento expone el Plan de Trabajo de la Estancia de Investigación
que realizará Jesús Adolfo Mejía de Dios, con matrícula S16017501, estudiante de la
Maestría en Inteligencia Artificial (MIA) de la Universidad Veracruzana (UV). Dicha estancia se realizará bajo la asesoría del Dr. Edgar Alfredo Portilla Flores, investigador del Centro de Innovación y Desarrollo Tecnológico en Cómputo (CIDETEC) del Instituto Politécnico Nacional (IPN). El periodo que vigencia de la estancia será del 1 de marzo al 30 de abril del año en curso.

\section{Objetivo General}
Analizar el desempeño del algoritmo \textit{Evolutionary Centers Algorithm} (ECA) diseñado en el Centro de Investigación en Inteligencia Artificial de la UV para resolver problemas de optimización altamente complejos de la vida real.

\section{Objetivos Particulares}
\begin{itemize}
\item Exponer el algoritmo ECA a estudiantes en CIDETEC para que conozcan y le encuentren utilidad a esta nueva alternativa.
\item Familiarizarme con problemas de Mecatrónica que se encuentran resolviendo en CIDETEC.
\item Revisión de la literatura referente a los problemas en cuestión.
\item Ajustar el algoritmo ECA para la resolver un problemas, es decir, encontrar los parámetros adecuados tales que ECA resuelva, de manera acertada, los problemas elegidos.
\item Analizar los resultados obtenidos y aplicar técnicas estadísticas que indique cuál es el desempeño el algoritmo en cada escenario.
\item Escribir un reporte explicando los resultados del análisis realizado después de la experimentación.
\end{itemize}

% \clearpage

\section{Cronograma de Actividades}

\begin{center}
	
\rowcolors{1}{gray!50}{}
\hspace{-1.0cm}
\begin{tabular}{|l|c|c|c|c|c|c|c|c|}
\hline \rowcolor{gray}
% 
&\multicolumn{8}{c}{\bf Número de Semana}\\ \hline
% 
%%%%%%%%%%%%%%%%%%%%%%%%%%%%%%%%%%%%%%%%%%%%%%%%%%%%%%%%%%%%%%%%%%%%%%
{\bf Actividad } 
&{\bf 1}&{\bf 2}&{\bf 3}&{\bf 4} &{\bf 5}&{\bf 6}&{\bf 7}&{\bf 8} \\ \hline
%%%%%%%%%%%%%%%%%%%%%%%%%%%%%%%%%%%%%%%%%%%%%%%%%%%%%%%%%%%%%%%%%%%%%%
Revisión bibliográfica de la problemática a resolver
& \tache  & \tache & \tache & \tache  & \tache & \tache & \tache & \tache \\ \hline
%%%%%%%%%%%%%%%%%%%%%%%%%%%%%%%%%%%%%%%%%%%%%%%%%%%%%%%%%%%%%%%%%%%%%%
Exponer Algoritmo ECA
& \tache  &        &        &         &        &        &        & \\ \hline
%%%%%%%%%%%%%%%%%%%%%%%%%%%%%%%%%%%%%%%%%%%%%%%%%%%%%%%%%%%%%%%%%%%%%%
Familiarizarme con problemas representativos  
&         & \tache & \tache  & & & & & \\ \hline
%%%%%%%%%%%%%%%%%%%%%%%%%%%%%%%%%%%%%%%%%%%%%%%%%%%%%%%%%%%%%%%%%%%%%%
Ajustar ECA para resolver dichos problemas
&          & \tache & \tache &           & & & & \\ \hline
%%%%%%%%%%%%%%%%%%%%%%%%%%%%%%%%%%%%%%%%%%%%%%%%%%%%%%%%%%%%%%%%%%%%%%
Diseño de experimentos
&&&          & \tache & \tache  &          & & \\ \hline
%%%%%%%%%%%%%%%%%%%%%%%%%%%%%%%%%%%%%%%%%%%%%%%%%%%%%%%%%%%%%%%%%%%%%%
Análisis de resultados y comprobación estadística
&&&&          &          & \tache & \tache &  \\ \hline
%%%%%%%%%%%%%%%%%%%%%%%%%%%%%%%%%%%%%%%%%%%%%%%%%%%%%%%%%%%%%%%%%%%%%%
Redacción de reporte
&&&&&          &          & \tache & \tache \\ \hline
%%%%%%%%%%%%%%%%%%%%%%%%%%%%%%%%%%%%%%%%%%%%%%%%%%%%%%%%%%%%%%%%%%%%%%
\end{tabular}
\end{center}

\vspace{1cm}
\begin{center}
	\textbf{Atentamente}
	\\[3cm]
	\begin{minipage}{.4\textwidth}
		\centering
		\rule{\textwidth}{1pt}\\
		Jesús Adolfo Mejía de Dios\\
		Interesado
	\end{minipage}
	\\[3cm]
	\begin{minipage}{.4\textwidth}
		\centering
		\rule{\textwidth}{1pt}\\
		Dr. Efrén Mezura Montes\\
		Asesor Académico\\
		Vo.Bo.
	\end{minipage}
	\hspace{1cm}
	\begin{minipage}{.4\textwidth}
		\centering
		\rule{\textwidth}{1pt}\\
		Dr. Edgar Alfredo Portilla Flores\\
		Asesor Anfitrión\\
		Vo.Bo.
	\end{minipage}

\end{center}

\end{document}



















